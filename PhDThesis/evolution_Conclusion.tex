\section{Conclusion}
\label{sec:conclusion}
In this chapter, we aimed to discover the archetypical research behavior of Academics. The observation that despite surface variation in terms of sub-fields, the change in behavior exhibits regularities, motivated our research. We introduced a novel Gaussian Hidden Markov Model Cluster (G-HMM) to identify archetypes and evolutionary patterns within each archetype. We chose to work with G-HMM's since they allow for: variations in trajectories and different evolutionary rates; constrain how individuals can evolve; are interpretable.

We identified four distinct archetypes of computer scientists: steady, diverse, evolving, and diffuse and showed examples of computer scientists from different sub-fields that share the same archetype. We analyzed full professors from the top 50 CS departments to understand gender differences within archetypes.
Women and men differ within an archetype (e.g., diverse) in where they start, rate of transition and research interests during mid-career. We further analyzed grant income of these professors to understand the effect of gender and archetype on income. The differences in income are salient across states within an archetype rather than across archetypes. There also exist significant differences across genders within a state of an archetype. We observed that most of the grant income variability is accompanied by a shift in the dominant research area of the academic. In light of our findings, we propose the funding agencies to be cautious of these differences when deciding on grant applications submitted by researchers venturing into new areas.

To the best of our knowledge, we are the first one to provide a principled framework to model and identify interpretable individual trajectories in academia. Our model can be easily used to identify trajectory in other domains like medicine, physics, and business. Further work on the comparison of research trajectories from the stem and non-stem fields could be an exciting research direction.

For StackOverflow, discovered archetypes could be labeled as: \emph{Experts}, \emph{Seekers}, \emph{Enthusiasts} and \emph{Facilitators}. We showed strong quantitative results with competing baselines for future activity prediction and perplexity.
