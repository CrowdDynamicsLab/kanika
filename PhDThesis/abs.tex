User behavior modeling has become an indispensable tool with the proliferation of socio-technical systems to provide a highly personalized experience to the users. These socio-technical systems are used in sectors as diverse as education, health, law to e-commerce, and social media.
The two main challenges for user behavioral modeling are building an in-depth understanding of online user behavior and using advanced computational techniques to capture behavioral uncertainties accurately.
This thesis addresses both these challenges by developing interpretable models that aid in understanding user behavior at scale and by developing sophisticated models that perform accurate modeling of user behavior.

Specifically, we first propose two distinct interpretable approaches to understand explicit and latent user behavioral characteristics.
Firstly, in \Cref{chap:evolution}, we propose an interpretable Gaussian Hidden Markov Model-based cluster model leveraging user activity data to identify users with similar patterns of behavioral evolution. We apply our approach to identify researchers with similar patterns of research interests evolution.  We further show the utility of our interpretable framework to identify differences in gender distribution and the value of awarded grants among the identified archetypes. We also demonstrate generality of our approach by applying on StackExchange to identify users with a similar change in usage patterns.

Next in \Cref{chap:reliability}, we estimate user latent behavioral characteristics by leveraging user-generated content (questions or answers) in Community Question Answering (CQA) platforms. In particular, we estimate the latent aspect-based reliability representations of users in the forum to infer the trustworthiness of their answers. We also simultaneously learn the semantic meaning of their answers through text representations. We empirically show that the estimated behavioral representations can accurately identify topical experts.

We further propose to improve current behavioral models by modeling explicit and implicit user-to-user influence on user behavior. To this end, in \Cref{chap:social}, we propose a novel attention-based approach to incorporate influence from both user's social connections and other similar users on their preferences in recommender systems. Additionally, we also incorporate implicit influence in the item space by considering frequently co-occurring and similar feature items. Our modular approach captures the different influences efficiently and later fuses them in an interpretable manner.
Extensive experiments show that incorporating user-to-user influence outperforms approaches relying on solely user data.

User behavior remains broadly consistent across the platform. Thus, incorporating user behavioral information can be beneficial to estimate the characteristics of user-generated content.
To verify it, in \Cref{chap:induced}, we focus on the task of best answer selection in CQA forums that traditionally only considers textual features. We induce multiple connections between user-generated content, i.e., answers, based on the similarity and contrast in the behavior of authoring users in the platform. These induced connections enable information sharing between connected answers and, consequently, aid in estimating the quality of the answer.
We also develop convolution operators to encode these semantically different graphs and later merge them using boosting.

We also proposed an alternative approach to incorporate user behavioral information by jointly estimating the latent behavioral representations of user with text representations in \Cref{chap:syntactic}. We evaluate our approach on the offensive language prediction task on Twitter. Specially, we learn an improved text representation by leveraging syntactic dependencies between the words in the tweet. We also estimate the abusive behavior of users, i.e., their likelihood of posting offensive content online from their tweets. We further show that combining the textual and user behavioral features can outperform the sophisticated textual baselines.
