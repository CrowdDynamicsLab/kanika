\section{Limitations and Future Work}
\label{sec:Limitations}

Our proposed model identified insightful archetypes and its variability with gender and grant income of professors. However, it is essential to understand certain caveats to the reported findings.
First, in terms of the data, the discovered archetypes for academics are for the top researchers in their field (we pick \emph{prominent} researchers in each of the 35 research subdomains) with a long career span (15 years). Thus, our archetypes do not reflect junior scientists engaged in research or researchers with sporadic research output. Nonetheless, our study will help junior scientists to understand diverse ways of research behavior of successful academics in the field and tailor their career to their interests.
In our current study, we collected grant history from public data available by NSF. The funding analysis can be extended by collecting data from other possible funding sources like the National Institute of Health (NIH), gifts from industry, and professor's salary. The findings of gender bias can be different if we include these private sources of income (\cite{Ward:2001} observed gender differences in professor's pay).
Hence, we believe that our study is the first step in understanding differences in research conducting behavior of academics and its effect on their income.

Second, as with all inductive models, our qualitative results depend on the chosen model. Recently, Deep Neural Networks, especially Recurrent Neural Networks, have been proposed to model time series data. There has also been considerable interest in building interpretable neural models \cite{Ribeiro:2016, hima:2016}. However, still neural approaches are hard to interpret, and developing interpretable neural prediction models is something that we plan to look at in future work.

Third, in our current version of the model, we do not consider the effect of collaborations or the role of conferences where researchers publish, and where they may pick up on normative behavior (e.g., areas in which to work) on the discovered archetype. In future work, we plan to understand the role of community interaction on archetypes and address these limitations. Another interesting research direction is to explore the correlation of change in research behavior with career transitions and author's citation count.
