\chapter{Literature Review}
\label{sec:related}

Before delving into details of our proposed approach, we first discuss prior literature related to User Behavior Modeling in Online Social Networks, Academic Dataset and Recommender Systems. We also review text representation approaches used in Community Question Answering (CQA) forums and for short text in Twitter. We also briefly review recently proposed Graph convolution networks to model graph-structured data (used in our work) for these platforms.

\section{Online Social Networks}
There has been a lot of interest in the past on identifying and characterizing user behavior in online social networks (OSNs). \citet{Maia:2008} identified five distinct user behaviors of YouTube users based on their individual and social attributes. While \citet{Mamykina:2011} identified user roles based on just answer frequency in StackExchange. \citet{Adamic:2008} and \citet{Furtado:2013} worked on similar user behavioral studies on Yahoo Answers and Stack Overflow datasets, respectively. All these studies, however, ignore \emph{temporal changes} in the behavior and use engineered features for behavior modeling.

Some behavioral studies do model evolution of user activities in the platform too. \citet{Benevenuto:2009} learned a Markov model to examine transition behavior of users between different activities in Orkut in a static snapshot. \citet{Yang:2014} and \citet{Knab2003} proposed generative models that assigned each user action to a progression stage and classify event sequences simultaneously. They used their model to predict cancer symptoms, or products user would review in the future. However, the model did little to provide meaningful and interpretable stages and clusters. \citet{Angeletou:2011} constructed handcrafted rules to identify user roles and studied the change of user roles' composition in the community over time. Recently, \citet{Santos:2019} identified four distinct types of user activity pattern based on their activity frequency.

The Hidden Markov Model (HMM) has been widely used to model and cluster time sequences \citep{Smyth:1997, Bicego:2003, Coviello:2014} in the past. However, most of these models learn an HMM for each user sequence and then employ clustering algorithms to cluster the learned HMMs. These approaches are not scalable, and the clusters thus identified are not interpretable.

\section{Recommender Systems}
In the following section, we provide a brief review of approaches that model users' historical interactions to improve recommender systems. We first enlist approaches assuming static user behavior (Collaborative Filtering). Consequently, we review approaches that model the evolution of user behavior (Temporal Recommendation), social influence (Social Recommendation), and few recently proposed methods which are looking at combining the two (Socio-Temporal Recommendation).

\noindent
\emph{Collaborative Filtering:}
Collaborative Filtering (CF) is one of the most popular techniques for user modeling in recommender systems.
Specifically, the methods employ Matrix Factorization (MF) to decompose a user-item rating matrix into user and item specific latent factors.
Classical and seminal work for MF-based recommender systems~\cite{Rendle} uses a Bayesian pairwise loss (BPR). Collaborative filtering is also performed in item space~\cite{itemCF}, where similar items are computed offline based on their rating similarity or co-occurrence in the dataset. Consequently, it recommends items similar to the ones used in the past by the user.
Neural net approaches have been proposed recently to improve MF models. They learn more complex non-linearities in the user-item interaction data~\cite{NeuMF, CDAE}.

However, most MF approaches assume a static user-item interaction matrix. Often, this assumption is not accurate, particularly for online communities where user preferences evolve over time --- sometimes quickly --- necessitating temporal recommendation.

\noindent
\emph{Temporal Recommendation:}
There has been significant work in the area of temporal recommender systems that model a user's past interactions to inform a user's current preference. These temporal models generally assume a linear relationship between the events and model it using a Markov chain~\cite{FPMC, Rendle2}. However, these are often `shallow' (i.e., linear) methods that are inept at modeling the more complex dynamics of temporal changes. Recent works~\cite{Sun:2018, Cai:2017, SAS:2018} use deep net approaches involving convolution layers, attention networks, and recurrent neural nets to model complex relations. For example,~\citet{Caser} applies convolutional filters on the embedding matrix computed from a few recent items of a user. This model captures a higher-order Markov chain, but it still has a limited scope as it does not consider the entire history of a user.
In contrast, to model long term dependencies,~\citet{GRU4Rec} propose to model a user's sequential behavior within a session using recurrent neural nets. \citet{RRN} apply a recurrent architecture to both user and item sequences and hence model dynamic influences in popularity of movies on users' viewing preference. \citet{SAS:2018} instead employ a self attention module for next item recommendation that adaptively learns the importance of all past items in a user's history. However, these models are limited as they do not leverage the social connections of a user.

\noindent
\emph{Social Recommendation:}
Social recommenders integrate information from a user's social connections to mitigate data sparsity for cold-start users, i.e., users with no or minimal history. They exploit the principle of social influence theory~\cite{Tang:2009}, which states that socially connected users exert influence on each other's behavior, leading to a homophily effect: similar preferences towards items. \citet{SocialMF, SoReg} use social regularization in matrix factorization models to constrain socially connected users to have similar preferences. The recently proposed SERec~\cite{SERec} embeds items seen by the user's social neighbors as a prior in an matrix factorization model. The SBPR model~\cite{SBPR} extends the pair-wise BPR model to incorporate social signals so that users assign higher ratings to items preferred by their friends. However, these models assume equal influence among all social neighbors. TBPR~\cite{TBPR} distinguishes between strong and weak ties only when computing social influence strength.

\noindent
\emph{Socio-Temporal Recommendation:}
Few of the recent approaches have started to look at merging temporal dependence with social influence. \citet{Cai:2017} extend Markov chain based temporal recommenders~\cite{Rendle2} by incorporating information about the last interacted item of a user's friends. This work assumes markov dependence i.e. the future item just depends on the current item. This assumption is limiting im modeling evolving user preferences.

In the context of session-based recommendation, \citet{Sun:2018} propose a socially aware recurrent neural network that uses a dynamic attention network to capture social influence. On the other hand, \citet{Song:2019} use graph attention nets to model social influence on a user's behavior in the session. Both these models learn a unified user representation based on social influence with a user's temporal history.


\section{Scholarly Data}
Most of the work on user behavioral mining concerns career movement within academia. \citet{deville:2014} observed that transitions between academic institutions are influenced by career stage and geographical proximity. While \citet{clauset:2015} found that academic prestige correlates with higher productivity and better faculty placement. Recently, \citet{Danai:2018} studied career transitions across academia, government, and industry for Computer Science researchers. \citet{dashun:2013} proposed a statistical model to predict the most impactful paper, in terms of citations, of scientists across disciplines. They argued nonexistence of a universal pattern and showed that highest-impact work in a scientist's career is randomly distributed within her body of work.

Recent studies also looked at gender differences in funding patterns, productivity, and collaboration trends in academia \citep{Way:2016, Way:2017}. \citet{Way:2016} did not observe any significant difference across gender in hiring outcomes in academia. However, they showed that indirect gender differences exist in terms of productivity, postdoctoral training rates, and in career growth. Some earlier studies also reported gender differences in academia. \citet{Kahn:1993} identified gendered barriers in obtaining tenure for academics in economics, while \citet{Ward:2001} found gendered differences in pay related to publication record.

There also has been considerable interest in mining scholarly data produced by researchers (bibliographic data, researchers' usage of social media, etc.). Prior studies have looked at the evolution of research interests on a community level. \citet{liu2014chi} studied the evolution of research themes in articles published in CHI conference on Human Computer Interaction through co-word analysis. They highlighted specific topics as popular, core, or backbone research topics within the community. While \citet{Biryukov:2010} compared different scientific communities in DBLP dataset in terms of its interdisciplinary nature, publication rates, and collaboration trends. They also studied the variation of author's productivity with career length and observed that most of the authors have a short career spanning less than five years. \citet{Chakraborty:2018} studied trajectories of successful papers in computer science and physics by analyzing paper citation counts. They classified these trajectories into multiple categories including early riser, a late riser, steady riser, and steady dropper.

\section{Community Question Answering Forums}
Community Question Answering forums are increasingly used to seek advice online; however, they often contain conflicting and unreliable information. This misinformation could lead to serious consequences to the users. Thus, most of the work that model user behavior in CQA forums deals with predicting user reliability or quality of posted answers to a question.

Prior works can be classified into Feature-driven models; which use user and content-based engineered features for the task; another is Deep Text models that only model relevance of the content of question and answers for prediction and disregard user information. Recently, unsupervised approaches based on Truth Discovery principle are applied to model user expertise and answer quality simultaneously in these forums.

\noindent
\emph{Feature-Driven Model:}
Feature-driven models \cite{BurelMA16} develop features from three different perspectives: user features, content features, and thread features.
These features are fed into classifiers, such as tree-based models \cite{BurelMA16, JendersKN16, TianZL13} to identify the best answer. \citet{TianZL13} found that the best answer is usually the earlier and most different one, and tends to have more details and comments. \citet{JendersKN16} trained several classifiers for online MOOC forums. Different from existing works, \citet{BurelMA16} emphasize on the thread-like structure of question \& answer and introduce four thread-based normalization methods. These models predict the answer label independently of the other answers for the question.
CQARank leverages voting information as well as user history and estimates user interests and expertise on different topics ~\cite{yang2013cqarank}. \citet{barron2015thread} also look at the relationship between the answers, measuring textual and structural similarities between them to classify useful and relevant answers. All these supervised approaches need a large amount of labeled training data ~\cite{wen2018hybrid, mihaylova2018fact,oh2013finding}. However, it is expensive and unsustainable to curate each answer manually for training these models. Alternatively, forums employ crowd sourced voting mechanisms to estimate information reliability but it could lead to under-provision \cite{gilbert2013widespread}.

\noindent
\emph{Deep Text Models:} Text-based deep learning models learn an optimal representation of question-answer text pairs suitable to select the best answer \cite{ZhangLSW17, WuWS18, WangN15}. In SemEval 2017 on Community Question Answering (CQA),~\cite{nakov2017semeval} developed a task to recommend useful related answers to a new question in the forum.
SemEval 2019 further extends this line of work by proposing fact checking in community question answering~\cite{Mihaylova2019semeval}. \citet{FengXGWZ15} augment CNN with discontinuous convolution for a better vector representation; \citet{WangN15} uses a stacked biLSTM to match question and answer semantics. \citet{SukhbaatarSWF15} use attention mechanism in an end-to-end memory framework. Text-based models take longer to train and are computationally expensive.

\noindent
\emph{Truth discovery:} Different approaches based on truth discovery principle have been proposed to address predict answer quality in CQA forums~\cite{zhang2018texttruth, li2015discovery, zheng2017truth,li2016crowdsourcing,mukherjee2016truthcore,vydiswaran2011content}. Many truth discovery approaches are tailored to categorical data and thus assume there is a single objective truth that can be derived from the claims of different sources \cite{li2016survey}. Faitcrowd~\cite{ma2015faitcrowd} assumes an objective truth in the answer set and uses a probabilistic generative model to perform fine-grained truth discovery. It jointly models the generation of questions and answers to estimate the source reliability and correct answer. On the other hand,~\citet{wan2016truth} propose trustworthy \emph{opinion} discovery where the true value of an entity is modeled as a random variable with a probability density function instead of a single value.

Some truth discovery approaches also leverage text data to identify correct responses better.~\citet{li2017reliable} proposed a model for capturing semantic meanings of crowd provided diagnosis in a Chinese medical forum. In particular, they use a medical-related dictionary to extract terms in the response text and learn their semantic representations to discover trustworthy answers from non-expert users in crowdsourced diagnosis.
\citet{zhang2018texttruth} proposed a Bayesian approach to capture the multifactorial property of text answers and used semantic representations of keywords to mitigate the diversity of words in answers. To model the user reliability, the authors proposed a two-fold reliability metric that uses both false positive and true positive rates. These approaches only use certain keywords for each answer and are thus, limited in their scope.


\section{Twitter}
Most previous methods for detecting offensive speech on Twitter rely entirely on the textual content.
Most of these prior work includes using statistical features like bag-of-words or tf-idf features for automated detection.\citet{wulczyn2017} used character n-gram features for detecting abusive comments in the discussion on Wikipedia pages. On Twitter dataset, \citet{waseem-hovy-2016} used character and word n-gram features along with lexical and users features to detect hate speech. \citet{davidson2017automated} worked with character n-grams on a different Twitter dataset to achieve competitive performance. On the other hand, \citet{nobata} combined n-grams features with linguistic, syntactic, and semantic features. However, they observed that n-gram features are most beneficial for the detection task.
Even though bag-of-words approaches perform well, they are unable to capture nuanced hate speech as they fail to contextualize the word meanings.
For instance, depending on the context, the word \emph{gay} can be used to denote either ebullience or sexual preference. Only the latter is a candidate attack.

Recently, deep learning models are also proposed that leverage pre-trained word embeddings such as word2vec \cite{mikolov2013distributed} and Glove \cite{glove}
to capture aspects of the semantics of the tweets. These models aggregate individual word embeddings in a context-aware manner to compute tweet embeddings and later use them for classification.
\citet{gamback} and \citet{park2017one} used the Convolutional Neural network to compute the tweet embeddings while \citet{badjatiya2017deep} and \citet{agrawal} showed that Gated Recurrent Units or Long-Short Term Memory networks are useful to compute these embeddings.
On the other hand, \citet{ziqicnn} used a combination of CNNs and GRU to achieve competitive performance.

The syntactic structure of the text can also be used to help identify the target group and the intensity of hate speech. For instance,~\citet{warner2012} extracts POS-based trigrams such as DT jewish NN to extract hate speech against a specific target, Jews. While, \citet{silva2016} extends it further to look for generic syntactic structures like "I $<$intensity$>$ hate $<$target$>$'.
The primary difficulty of this work is that the space of possibly relevant rules is too large for an analyst to be confident that the list is truly comprehensive.
In addition, it verges on the impossible to specify a set of rules that will do a decent job on the endless variety of possible implicit attacks.

A minority of approaches take advantage of non-textual user data in addition to the text.
\citet{2017improved} added randomly-initialized user embeddings to their RNN model to obtain higher accuracy.
\citet{qian2018} showed that incorporating intra-user and reinforced inter-user representations significantly improve the performance of their bi-directional LSTM model. However, both of these approaches work on the individual user level and ignore the social influence on their behavior. \citet{mishra2018} captured the social influence in abusive accounts by computing a representation of a user's neighborhood through node2vec features.  The classifier described in \citet{mishra2019abusive} extends the previous paper by computing a user representation from an extended graph of users and tweets.


\noindent
\section{Graph Convolution Networks}
More recently, Graph Convolution Networks (GCNs) have been proposed to learn embeddings for graph-structured data~\cite{Kipf:2016}. Graph Convolution can be applied in both spatial and spectral domains to compute node representations. The learned node representations are then used for various downstream tasks like node classification \cite{gcn}, link prediction \cite{relationalGCN}, multi-relational tasks \cite{rase} etc. Spatial approaches employ random walks or k-hop neighborhoods to compute node representations  \cite{DeepWalk, node2vec, Planetoid, LINE}. Pioneer works on graph convolution in the spectral domain use fast localized convolutions ~\cite{deferrard, duvenaund}. Recently proposed Graph Convolution Networks~\cite{gcn} outperforms spatial convolutions and are scalable to large graphs. Various extensions to the GCN model have been proposed for signed networks \cite{signedgcn}, inductive settings \cite{graphsage} and multiple relations \cite{DualGCN, relationalGCN} and evolution~\cite{dysat}. All of the GCN variants assume label sharing as they assume similarity between connected nodes.

In Recommender Systems, GCNs have been used to model the user-item interaction graph. GCMC~\cite{GCMC} extends GCN by training an auto-encoder framework on a  bipartite user-item interaction graph that performs differentiable message passing, aggregating data from a user's and an item's 'neighbors'. PinSage~\cite{PinSage} proposed a random walk based sampling of neighbors to scale GCNs to web scale graphs. \citet{fan2019} further extend these methods to incorporate information from a user's social connections.
Similarly, \citet{Diffnet} use graph neural networks to model diffusion of social influence in recommender systems.

However, these methods either do not take a user's social neighbors into account or operate on static features.  All these models also assign uniform weight to all their neighbors, which does not represent online social communities well. Typically in these communities, some friends are only superficially known while others are known personally for years. Thus, they exert a different degree of influence on a user's behavior. Graph Attention Networks~\cite{GAT} can capture the varying influence stengths as they learn attention weights between each pair of nodes in a static graph.
