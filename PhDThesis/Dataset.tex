\section{Data Collection}
\label{sec:dataset}

We use the Microsoft Academic dataset \cite{Sinha:2015} provided through their Knowledge Service API\footnote{\url{http://bit.ly/microsoft-data}} to study evolutionary patterns of researchers with a focus on Computer Scientists.
Microsoft Academic Service additionally annotates each publication with the year of publication, publication venue and the CS subfield (out of $35$ identified fields) to which it belongs.

To get sufficient data for our proposed generative model, we decide to identify \emph{influential} scientists with consistent publication history.  This author data will help us discover \emph{prominent} archetypes of research interest evolution in the field. We identify \emph{influential} authors based on \emph{prominence} of the conference venues in which they publish. To quantify \emph{prominence} of a conference, we construct a conference-conference citation graph where each conference in our dataset forms a node, and the weighted edges represent inter-conference citation frequency. Specifically, the weight of a directed edge from conference $C_1$ to conference $C_2$ is proportional to the fraction of papers published in $C_2$ cited by papers published in $C_1$. We then use the Pagerank algorithm \citep{ilprints422} on this directed graph and define conference \emph{prominence} as the Pagerank of the corresponding conference-node. After that, we define an author's \emph{influence} as the weighted sum of the prominence of the conferences (s)he has published in. Here, conference-prominence are further weighted by the fraction of the author's papers published in that venue.


We rank authors in decreasing order of their \emph{influence} in each of the $35$ CS areas (as annotated by Microsoft API) in the dataset. To get equal representation from all subareas, we then extract the publication history of top $750$ most-influential authors from each of the subareas in the dataset. Note that authors can be \emph{influential} in more than one subfield. We then filter unique authors from this set who have at least 15 years of publication history. We only consider publication history from 1970 to 2016 to avoid missing data. The resulting dataset consists of records of $4578$ authors with an average publication history of 24.15 years. \footnote{This data will be made available upon publication.}

We now describe in detail how we represent an author's academic life-cycle as a sequence, $\mathbf{X_i}$, comprising of session-vectors, $X_{ij}$. We use a normalized representation for session-vectors focused on the change of areas. Specifically, we represent a session-vector as the fraction of papers an author publishes in various \emph{area-of-interests} (\texttt{AoI}s) in that year.

For defining an \texttt{AoI} of an author, we consider all papers published by the author in her academic life. We identify her primary \texttt{AoI}, $D_1$, as the \emph{first} subfield (out of 35 subfields) in which she publishes \emph{cumulatively} at least $3$ papers in the first 3 years. Usually, an author's $D_1$ is about their Ph.D. dissertation work, and we expect students to \emph{settle} down after a few years. Thus, after identification of $D_1$, hopefully with a steady paper count, we define her secondary \texttt{AoI}, $D_2$, as the subfield in which she publishes at least $3$ papers in \emph{one} year. Similarly, we also define tertiary ($D_3$), quaternary ($D_4$), and quinary ($D_5$) \texttt{AoI}. We do not define \texttt{AoI}s beyond $D_5$ because 80\% of authors do not explore more than $5$ subfields in our dataset. Also, in a given year, if an author publishes fewer than $3$ papers in an unexplored subfield, these papers count towards a sixth dimension \texttt{AoI} called \emph{Explore} (Ex). \emph{Explore} dimension denotes that the author has started exploring new subfields but are not notable enough to be one of the $D_m$'s ($m \in {[1,5]})$, and indicate a possible shift in research interests.

To summarize, each session is a $6$ dimensional vector ($M=6$), and its elements are fraction of the author's publications in the $5$ $D_m$'s or the $6^{th}$ \emph{Explore} dimension. This normalized session representation allows our model to discover behavioral patterns of the author's changing research interests in a domain-independent manner. For example, in a given year, the session-vector for an author who publishes 3 papers in theory ($D_1$; primary area) and 1 paper in graphics ($D_2$; secondary area), and the session-vector for another author who publishes 3 and 1 papers in NLP ($D_1$; primary area) and ML ($D_2$; secondary area) respectively will be exactly same: $X_{ij} = \langle 0.75, 0.25, 0, 0, 0, 0\rangle$. Notice that normalization does not change the rate at which a specific author decides to switch domains and is also invariant to subarea publication norms.
